\chapter{Software Architecture Design}
\label{chap:software-architecture-design}

\section{Domain Model}
\label{section:domain-model}

\begin{figure}[h!]
    \centering
    \includesvg[width=1\textwidth]{kueater/domain_model.svg}
    \caption{Domain Model of KU Eater}
    \label{fig:domain-model}
\end{figure}

The domain model of KU Eater (Figure \ref{fig:domain-model}) shows the business knowledge, any technicalities are abstracted into their most simplest form. (Recommendation Engine for example)

However, some clarification may be needed as to help readers understand:

\begin{itemize}
    \item Stall can aggregate reviews from MenuItem directly, so no need to associate Review with Stall.
    \item DietaryRestriction is considered separated from UserProfile. The benefit is that DietaryRestriction can be reused for other users
    and we can specify common personal diets like: Vegetarian, Halal etc.
\end{itemize}

\section{Design Class Diagram}
\label{section:design-class-diagram}

\begin{figure}[h!]
    \centering
    \includesvg[width=\textwidth,height=0.7\textheight,keepaspectratio]{kueater/class_diagram.svg}
    \caption{Class Diagram of KU Eater}
    \label{fig:class-diagram}
\end{figure}

Class diagram (Figure \ref{fig:class-diagram}) represents a structure of application that relates closely to real implementation. The diagram
shows mainly the Data Objects that KU Eater must keep track of. Class Diagram is also expanded upon the Domain Model (see Figure \ref{fig:domain-model})
making sure that the integrity of business knowledge is intact.

\section{Sequence Diagram}
\label{section:sequence-diagram}

Sequence diagrams visualize interaction between the user and the system. The figures below show how components in the system will interact with
each other and the end-user, while the users can dispatch messages or inputs into KU Eater.

\subsection{Default Recommendation Process}
\begin{figure}[h!]
    \centering
    \includesvg[width=\textwidth,height=0.4\textheight,keepaspectratio]{kueater/sequence_diagram_anonymous_recommendations.svg}
    \caption{Sequence Diagram for Default Recommendation Process}
    \label{fig:seq-default-recommendation-process}
\end{figure}

Figure \ref{fig:seq-default-recommendation-process} represents a sequence diagram for a process when a guest user is browsing KU Eater. When a user
starts to use KU Eater, the controller in the application layer will invoke the machine learning layer without specific parameters and generate new recommendations.

Once the recommendations are finished generating, the application requests list of recommendations and renders them back to user.

\subsection{User Recommendation Process}
\begin{figure}[h!]
    \centering
    \includesvg[width=\textwidth,height=0.4\textheight,keepaspectratio]{kueater/sequence_diagram_user_recommendations.svg}
    \caption{Sequence Diagram for Authenticated User Recommendation Process}
    \label{fig:seq-user-recommendation-process}
\end{figure}

Figure \ref{fig:seq-user-recommendation-process} represents a sequence diagram for a process when authenticated user logs on to browse on KU Eater.
When a user, starts to use KU Eater, the controller in the application layer will fetch the user profile from the domain layer and pass the attributes in the profile to
the machine learning layer with user profile as the parameters. The recommendation engine will generate a list of possible recommendations.

Once the recommendations are finished generating, it is re-ranked, filtered and evaluated by considering the properties in user profile: Liked menu items, reviewed menu items, disliked menu items and dietary restrictions. The final set of recommendations is rendered back to user.

%\section{Algorithm}
%\label{section:algorithm}
%<TIP: Optional, If you are working on a research project that proposes a new
%algorithm, you can describe your algorithm here. It can be in the form of
%pseudocode or any diagram that you deem appropriate./>

\newpage

\section{Pipeline}
\label{section:pipeline}

\subsection{Modelling the Recommendation Engine}
\begin{figure}[h!]
    \centering
    \includesvg[height=0.3\textheight,keepaspectratio]{kueater/algo_modelling.svg}
    \caption{Pipeline for Modelling the Recommendation Engine}
    \label{fig:pipeline-for-modelling}
\end{figure}

Our recommendation engine uses content-based filtering and item-based collaborative filtering
with K-nearest neighbor (kNN) as the backbone of the engine. \cite{singhanddwivedi:2023}
The model needs to be fed relevant data such as Menu Categories, Menu Items and Stalls in which they exist in
the database. As shown in, figure \ref{fig:pipeline-for-modelling}.

The following data types are used in the modelling process:

\begin{enumerate}[leftmargin=80pt]
    \item \textbf{Menu items:} A structured data type containing menu name, ingredients and cooking method.
    \item \textbf{Menu categories:} A set of keywords that corresponds to menu item types, cuisines, cooking method etc.
    The categories are used to tag the menu items to further enhance decision making for model and simplify its
    similarity calculation.
    \item \textbf{Stalls:} A structured data type containing stall name, menu items a stall offers etc.
    It is used to multiply menu entries to a number of stalls, so that we can further score specific item from reviews.
\end{enumerate}

The main objective of this recommendation system is to figure out what food items are relevant with inputted food items that user interacted with.
In order to build the proper system, we must start with finding relevance between menu items by using \textit{Content-Based Filtering}.

\begin{enumerate}[leftmargin=80pt]
    \item We compile a table of menu items which has the following attributes, menu name, a list of ingredients and cooking method as a string.
    \item The tags from Menu categories will automatically populate within each menu item.
    \item We use cosine similarity technique to match any string-based attributes.
    \item We turn ingredients feature into binary (Exists, Non-exists), in that way we can score the similarity of food items with existence of each ingredients.
    \item Finally, we can then use cosine similarity measure to find distances between all food items.
\end{enumerate}

After having a precomputed similarity vectors of each food item, we can use them to determine recommendations for a user;
by using \textit{Item-Based Collaborative Filtering}.

\begin{enumerate}[leftmargin=80pt]
    \item Once a user required new recommendations, a list of food items and stall is used to generate all possible recommendations.
    \item Using the list of similarity scores from CBF, we can determine the most similar menu items from user preferences with cosine similarity.
    \item We can take ratings of observed menu items and construct a user-item correlation set.
    \item Using cosine similarity, we can determine the nearest neighbors which are other users with similar preferences.
    \item We can use those other users' profile to see what menu item they're most likely to interact with, and use those as recommendations.
\end{enumerate}

\subsection{Predicting the Recommendations}
\begin{figure}[h!]
    \centering
    \includesvg[width=\textwidth,height=0.2\textheight,keepaspectratio]{kueater/algo_prediction.svg}
    \caption{Pipeline for Predicting Recommendations}
    \label{fig:pipeline-for-predicting}
\end{figure}

User profile contains information that can be used to predict their preferences: Items that user recently looked, items reviewed by user, items liked by user.
These properties are used in the recommendation engine to predict an unordered list of possible recommendations.

However, the recommendations are not refined towards user, so additional properties are used externally: Disliked items and dietary restrictions.
"Disliked items" is used to reduce the appearance of disliked items, while "Dietary restrictions" is used to remove items from the list that users
cannot have. As shown in, figure \ref{fig:pipeline-for-predicting}.

\section{Schemas}
\label{section:data-schema}

KU Eater requires a lot of data from many sources as mentioned in \ref{subsection:data-acquisition}. In order to standardize and streamline the
data acquisition and processing phases, schemas for each important data type must be clear. The following are the schemas for specific data types:

\subsection{Menu Item}
\label{schema:menu-item}
\begin{verbatim}
    menu_id: uid,
    
    menu_name: string,

    price: double,
    
    ingredients: int<Ingredient>[],
    # refers to ingredients in database.
    
    tags: int[]
    # refers to menu categories in database.
\end{verbatim}

\subsection{Ingredient}
\label{schema:ingredient}
\begin{verbatim}
    ingredient_id: uid,
    
    ingredient_name: string
\end{verbatim}

\subsection{Stall}
\label{schema:stall}
\begin{verbatim}
    stall_id: uid,

    stall_name: string,

    items: uid<MenuItem>[]
    # refers to what items they are selling.
\end{verbatim}

\subsection{User Profile}
\label{schema:user-profile}
\begin{verbatim}
    profile_id: uid,
    
    name: string,
    
    liked: uid<MenuItem>[],
    # refers to liked menu items.

    disliked: uid<MenuItem>[],
    # refers to disliked menu items.

    restrictions: DietaryRestriction
\end{verbatim}

\subsection{Dietary Restriction}
\label{schema:dietary-restriction}
\begin{verbatim}
    blacklist: uid<Ingredient>[]
    # refers to ingredients prohibited
    for consumption.
\end{verbatim}