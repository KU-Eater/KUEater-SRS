\chapter{Software Architecture Design}
\label{chap:software-architecture-design}

\section{Domain Model}
\label{section:domain-model}

\begin{figure}[h!]
    \centering
    \includesvg[width=1\textwidth]{kueater/domain_model.svg}
    \caption{Domain Model of KU Eater}
\end{figure}

The domain model of KU Eater shows the business logic, any technicalities are abstracted into their most simplest form. (Recommendation Engine for example)
However, some clarification may be needed as to help readers understand:

\begin{itemize}
    \item Stall can aggregate reviews from MenuItem directly, so no need to associate Review with Stall.
    \item DietaryRestriction is considered separated from UserProfile. The benefit is that DietaryRestriction can be reused for other users
    and we can specify common personal diets like: Vegetarian, Halal etc.
\end{itemize}

\section{Design Class Diagram}
\label{section:design-class-diagram}

\begin{figure}[h!]
    \centering
    \includesvg[width=\textwidth,height=0.7\textheight,keepaspectratio]{kueater/class_diagram.svg}
    \caption{Class Diagram of KU Eater}
\end{figure}

\section{Sequence Diagram}
\label{section:sequence-diagram}

\subsection{Overall Recommendations Process}
\begin{figure}[h!]
    \centering
    \includesvg[width=\textwidth,height=0.4\textheight,keepaspectratio]{kueater/sequence_diagram_anonymous_recommendations.svg}
    \caption{Sequence Diagram for Overall Recommendations Process}
\end{figure}

%\section{Algorithm}
%\label{section:algorithm}
%<TIP: Optional, If you are working on a research project that proposes a new
%algorithm, you can describe your algorithm here. It can be in the form of
%pseudocode or any diagram that you deem appropriate./>