\chapter{Software Architecture Design}
\label{chap:software-architecture-design}

\section{Domain Model}
\label{section:domain-model}

\begin{figure}[h!]
    \centering
    \includesvg[width=1\textwidth]{kueater/domain_model.svg}
    \caption{Domain Model of KU Eater}
    \label{fig:domain-model}
\end{figure}

The domain model of KU Eater (Figure \ref{fig:domain-model}) shows the business knowledge, any technicalities are abstracted into their most simplest form. (Recommendation Engine for example)

However, some clarification may be needed as to help readers understand:

\begin{itemize}
    \item Stall can aggregate reviews from MenuItem directly, so no need to associate Review with Stall.
    \item DietaryRestriction is considered separated from UserProfile. The benefit is that DietaryRestriction can be reused for other users
    and we can specify common personal diets like: Vegetarian, Halal etc.
\end{itemize}

\section{Design Class Diagram}
\label{section:design-class-diagram}

\begin{figure}[h!]
    \centering
    \includesvg[width=\textwidth,height=0.7\textheight,keepaspectratio]{kueater/class_diagram.svg}
    \caption{Class Diagram of KU Eater}
    \label{fig:class-diagram}
\end{figure}

Class diagram (Figure \ref{fig:class-diagram}) represents a structure of application that relates closely to real implementation. The diagram
shows mainly the Data Objects that KU Eater must keep track of. Class Diagram is also expanded upon the Domain Model (see Figure \ref{fig:domain-model})
making sure that the integrity of business knowledge is intact.

\section{Sequence Diagram}
\label{section:sequence-diagram}

Sequence diagrams visualize interaction between the user and the system. The figures below show how components in the system will interact with
each other and the end-user, while the users can dispatch messages or inputs into KU Eater.

\subsection{Overall Recommendations Process}
\begin{figure}[h!]
    \centering
    \includesvg[width=\textwidth,height=0.4\textheight,keepaspectratio]{kueater/sequence_diagram_anonymous_recommendations.svg}
    \caption{Sequence Diagram for Overall Recommendations Process}
    \label{fig:seq-recommendation-process}
\end{figure}

Figure \ref{fig:seq-recommendation-process} represents a sequence diagram for a process when a user is browsing KU Eater. When a user
starts to use KU Eater, the controller in our application layer will invoke the machine learning layer and generate new recommendations.

Once the recommendations are finished generating, the application requests list of recommendations and renders them back to user.

%\section{Algorithm}
%\label{section:algorithm}
%<TIP: Optional, If you are working on a research project that proposes a new
%algorithm, you can describe your algorithm here. It can be in the form of
%pseudocode or any diagram that you deem appropriate./>

\newpage

\section{Pipeline}
\label{section:pipeline}

\subsection{Modelling the Recommendation Engine}
\begin{figure}[h!]
    \centering
    \includesvg[height=0.3\textheight,keepaspectratio]{kueater/algo_modelling.svg}
    \caption{Pipeline for Modelling the Recommendation Engine}
    \label{fig:pipeline-for-modelling}
\end{figure}

Our recommendation engine uses item-based collaborative filtering
with K-nearest neighbor (kNN) as the backbone of the engine. \cite{singhanddwivedi:2023}
The model needs to be feeded relevant data such as Menu Categories, Menu Items and Stalls in which they exist in
the database. As shown in, figure \ref{fig:pipeline-for-modelling}.

The following data types are used in the modelling process:

\begin{enumerate}[leftmargin=80pt]
    \item \textbf{Menu items:} A structured data type containing menu name, ingredients and cooking method.
    \item \textbf{Menu categories:} A set of keywords that corresponds to menu item types, cuisines, cooking method etc.
    The categories are used to tag the menu items to further enhance decision making for model and simplify its
    similarity calculation.
    \item \textbf{Stalls:} A structured data type containing stall name, menu items a stall offers etc.
    It is used to multiply menu entries to a number of stalls, so that we can further score specific item from reviews.
\end{enumerate}

\newpage

\subsection{Predicting the Recommendations}
\begin{figure}[h!]
    \centering
    \includesvg[width=\textwidth,height=0.2\textheight,keepaspectratio]{kueater/algo_prediction.svg}
    \caption{Pipeline for Predicting Recommendations}
    \label{fig:pipeline-for-predicting}
\end{figure}

User profile contains information that can be used to predict their preferences: Items that user recently looked, items reviewed by user, items liked by user.
These properties are used in the recommendation engine to predict an unordered list of possible recommendations.

However, the recommendations are not refined towards user, so additional properties are used externally: Disliked items and dietary restrictions.
"Disliked items" is used to reduce the appearance of disliked items, while "Dietary restrictions" is used to remove items from the list that users
cannot have. As shown in, figure \ref{fig:pipeline-for-predicting}.